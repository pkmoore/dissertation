\chapter{The SEA Technique and CrashSimulator}
\label{chap:sea}

In  this chapter, we document the development and implementation of a new
approach to finding and preserving anomalies that we call \textit{Simulating
Environmental Anomalies} (SEA). This technique is founded upon the key
insight that problematic environmental properties can often be detected
in the function calls, system
calls, or other interactions an application makes within an
environment. When employing SEA,
an application under test is exposed
to the anomalies unique to a given environment
in such a way that its responses will indicate
potential for failures upon deployment. In this way, developers are given
an easy and inexpensive way to learn from the mistakes of others, and
thus save money and programming hours that otherwise would be spent to
find and fix environmental bugs.

We found SEA capable of finding bugs,
both known and unknown,
in Linux applications ranked
highly on Debian's popularity contest
by implementing it in a proof of concept tool
called {\em CrashSimulator}\footnote{Our approach is
loosely inspired by flight simulators, which test pilot aptitude under a
variety of rare, adverse scenarios (water landings, engine failures,
etc.) before the pilots are certified to work in practice.}
and evaluating its performance~\cite{DebPopCon}.
These applications were chosen
because they are commonly used,
well tested,
and stable, giving CrashSimulator a chance to find new
environmental bugs where it should be the hardest.
Findings from these evaluations included bugs
attributed to
unanticipated file system configurations, file types, and network delays,
and resulted in a variety of failures, including hangs, crashes, and
filesystem damage.  In total, the SEA technique was able to identify 65
bugs with much less
time and effort than would be required to set up real environments and
execute the same applications within them -- illustrating SEA's
usefulness for developers in a real-world settin.

The main contributions in this work can be summarized as follows:

\begin{itemize}

\item{It provides evidence
that previously unanticipated flaws can be created by the interaction
between an application and its environment.}

\item{It introduces \textit{Simulating Environmental Anomalies} (SEA)
as an easy-to-use method for simulating environments
so an application's behavior in those environments
can be assessed before deployment---
without the time and resource costs of
testing in each environment.}

\item{It allows developers to build a corpus of extracted anomalies and thus
increase their capability to test applications against
problematic environmental aspects without per application effort.}

\item{It demonstrates a new tool, {\em CrashSimulator},
which implements SEA
in order to find previously-undiscovered environmental bugs
in widely deployed and highly tested code.}

\item{It introduces a new technique called {\it process set cloning}
that can generate copies of a running application,
so that users can test debugging hypotheses without damaging the
original.}

\item{It proves the effectiveness
of {\em CrashSimulator}
by showing it can find real bugs in real applications
when used by developers both involved and not involved with the project,
including developers with limited Linux experience.}

\end{itemize}
