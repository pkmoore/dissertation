\section{Related Work}
\label{SEC:related-work}

One of the ultimate goals of developing PORT
was to make it easier for developers to
create tools capable of
conducting program-level testing.
To design such a language,
we consulted
previous work
in processing sequences of events, such as
system calls, RPC invocations or
web-browser events.
Below, we discuss some of the more significant work in these areas.

\paragraph{System Call Stream Processing Applications.}

System call based intrusion detection systems fall into two categories: misuse and anomaly detection.
The former search for known patterns of application-specific
system call
sequences known as intrusion signatures~\cite{GARCIATEODORO200918}, while
the latter assumes that
any deviations
from ``normally observed'' system call sequences are
malicious~\cite{DBLP:conf/sp/ForrestHSL96}.

Forrest et al.~\cite{DBLP:conf/sp/ForrestHSL96} proposed 
an anomaly detection system that cataloga  witnessed patterns  within a database.
An application's system call stream
is monitored and any
deviation triggers a
predefined security policy.

%The automaton is not limited to recognizing the small system call sequence sizes
%proposed in~\cite{DBLP:conf/sp/ForrestHSL96}.
%Instead, the automaton learns the entire sequence of system calls produced by
%each run of the application.
%To help minimize the size of the automaton, the authors incorporate program
%counter information to recognize loops.
%In addition, system calls made within standard libraries (such as libc) are excluded from the automaton as the authors felt that these system calls do not necessarily help capture the unique nature of system call behavior within the program.

Ko et al.~\cite{DBLP:conf/acsac/KoFL94} 
proposes converting each system call in a
stream to a standard audit-policy record format that can be 
matched against program policy. 
However, the audit-policy can only be applied to
one system call at a time,
and does not support rules to recognize specific chains of system calls.
Another alternative is
Systrace~\cite{DBLP:conf/uss/Provos03},
which uses an associated policy language 
to describe any action prescribed when a rule evaluates to true.
Phoebe~\cite{DBLP:journals/corr/abs-2006-04444}
identifies patterns of system call failures during normal program execution
to test the reliability of an application when a failure occurs.
The downside is that more elaborate fault-injection
tests cannot be generated from these sequences.

Remote procedure calls can also be abused for malicious
intent,
so Giffin et al. ~\cite{DBLP:conf/uss/GiffinJM02} used
push-down automata to model the possible valid
remote call streams that an application might generate.
The application's incoming stream 
is then vetted
to determine whether particular calls are valid and therefore executable.

Lastly, there are some domain-specific options for
identifying problems
in function calls.
Christakis et al.~\cite{DBLP:conf/icse/ChristakisEG017} describe a language that allows developers to intercept and modify
Windows applications’ dynamic link library function calls to identify which ones should be
intercepted by the runtime.

%A commonality
%of all the systems
%outlined above
%is that they were built to solve one particular problem
%and
%therefore lack the flexibility that drove the creation of PORT.
%For example, these systems were not designed to be easily re-targeted
%to a different type of event stream or to allow for transformations.
%
It is likely that the previously cited FSA-based programs can be improved by applying recent advances in inference modeling algorithms~\cite{MarianiPS17,WalkinshawTD13,EmamM18,BeschastnikhBEK14}. Yet, these algorithms lack the conciseness and flexibility found in PORT. PORT does not require training sets and is expressive enough to specify both frequent and  “needle in the haystack” event sequences with just a few lines of code.


\paragraph{Event Stream Processing Languages and Algorithms.}
PORT can be categorized as a stream processing language,
which means it is domain-specific and
designed for expressing streaming applications.
In this section we look at previous work in this area

Pattern matching
over event streams is a paradigm
that looks for
possible matches against a previously defined set of rules. Collectively, these matches form a pattern.
Languages written for this purpose are significantly richer than those used for regular expression
matching~\cite{DBLP:conf/sigmod/AgrawalDGI08},
and typically provide automatic
support for naming, type checking, filtering, aggregating, classifying and
annotation of incoming events. They also  provide many benefits over traditional
stream-based text processing languages, such as sed~\cite{Mcmahon1979sed} and
awk~\cite{DBLP:journals/spe/AhoKW79}.

Though PORT is a stream processing language, it does not
require all of the features typically
included in this sort of system~\cite{DBLP:journals/csur/DayarathnaP18}.
Rather PORT seems to fit within the special case
known as complex event processing (CEP) 
Data items in input streams of these systems are referred to as raw events, while items in output streams are called
composite (or derived) events. A CEP system uses patterns to inspect
sequences of raw events and generate a composite event for each
match~\cite{DBLP:journals/ibmrd/HirzelAGJKKMNSSW13}

Queries and transforms written for CEP systems are
frequently compiled to a low-level general purpose language (C, C++, etc.) to allow for fast
processing of the stream. During the compilation process, automata are typically
built to recognize the patterns specified by the queries.

MatchRegex~\cite{DBLP:conf/debs/Hirzel12} is a CEP engine for IBM’s Stream Processing
Language. Predicates defined on the individual events appearing in the
stream can be utilized in the regular expression-based pattern matching
engine. MatchRegex supports regular expression operators, such as “Kleene star”
and “Kleene plus” over patterns consisting of predicates (boolean expressions).

GraphCE~\cite{DBLP:conf/models/BarqueroBTV18} describes the implementation of a CEP-like system on graph-based data. The system uses Scala code for pattern description but recommends the development of a DSL for use by data domain experts.
David et al. cover methods for dynamically modifying the underlying CEP query recognition model as the stream is being processed~\cite{DBLP:journals/sosym/DavidRV18}.

Though these CEP systems are capable
of recognizing the same stream patterns as PORT, they
do not incorporate the
transformation primitives 
required by the applications
envisioned for PORT. CEP systems are meant
to be used solely to recognize additional patterns.
It is the combination and interplay of pattern matching and transformation
primitives that distinguishes PORT from CEP systems.

\vspace{-.5cm}

%%% Local Variables:
%%% mode: latex
%%% TeX-master: "paper"
%%% End:
