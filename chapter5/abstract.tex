\abstract{Earlier work has proven that information extracted from recordings of an
application’s activity can be tremendously valuable. However, given the many requests that pass between applications and external entities, it has been difficult to isolate the handful of patterns that indicate the potential for failure. In this paper we propose a method that harnesses proven event processing techniques to find those problematic patterns. The key addition is PORT,
a new domain specific language which, when combined with its event stream
recognition and transformation engine,
that enables users to  extract patterns in system call recordings and other streams,
and then rewrite input activity on the fly.
The former task can spot activity that indicates a bug,
while the latter produces a modified stream
for use in more active testing.
We tested PORT's capabilities in several ways, starting with recreating the mutators and checkers utilized by an earlier work called SEA to modify and replay the results of system calls. Our
 re-implementations achieved the same
efficacy and better reliability using fewer lines of
code.
We also  illustrated PORT’s extensibility
by adding support for
detecting malicious USB commands
within recorded traffic.
}