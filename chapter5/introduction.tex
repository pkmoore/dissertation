\section{Introduction}
\label{sec:introduction}

%H. Dornhackl, K. Kadletz, R. Luh and P. Tavolato,
% "Defining Malicious Behavior,"
%2014 Ninth International Conference on Availability,
%Reliability and Security, Fribourg, Switzerland, 2014,
%pp. 273-278, doi: 10.1109/ARES.2014.43.
% Using a formal model to encode some sort of desired behavior

{\textit ``Actions speak louder than words...'' - Unknown}


It is a well established principle
that, in the wake of an application failure,
its actions
during execution can provide clues
to the root cause.
Such information
can not only help correct
the cause of failure,
but also prevent its repetition through the creation
of better test methods.
The challenge is 
how to identify and extract this data
from large and detailed sources like application logs,
system call traces,
or application recordings.
In other words, how does one
accurately describe what activity is important
and what you should do when you find it?
% This paper covers our efforts to make this process easier.

In considering this question,  we drew inspiration from two sources. The
first is a recent study that confirmed the value of monitoring and
modifying an application’s  interactions with its environment~\cite{DBLP:conf/issre/MooreCFW19}.  Using a technique known as SEA (Simulating Environmental Anomalies), the study demonstrated that when an application fails, the causal properties
will be visible in the results of the system calls it made. Further, the study affirmed these results could
be captured and simulated for testing against other applications.  The
second source was the significant amount of literature supporting the use of event
processing techniques over large streams of data~\cite{DBLP:conf/sigmod/AgrawalDGI08,DBLP:conf/debs/Hirzel12,DBLP:journals/ibmrd/HirzelAGJKKMNSSW13,DBLP:journals/csur/DayarathnaP18}. We posited that techniques
currently used to identify problems in  manufacturing environments, or patterns in
network outages, could also be used to accurately recognize target
sequences in large
application activity streams.

Building upon these successes,
we introduce a tool
that utilizes event processing techniques
to identify
behaviors 
that may cause applications to fail.
What makes this possible is PORT
(\textbf{P}attern \textbf{O}bservation, \textbf{R}ecognition, and
\textbf{T}ransformation),
a new domain specific language
we designed with the goal of
describing
these behaviors
in a briefer and more easily understood manner than conventional languages.
The descriptions can be used
to search recordings of an application's actions
across a variety of ``activity representations.'' That is, it can search activities like
system calls
or remote procedure calls and determine if  an application either executed a desired behavior
or avoided an undesired one.
Further, PORT can specify
a set of modifications
to be made
if a particular activity sequence is encountered.
By combining passive monitoring and active activity modification,
PORT can aid in identifying bugs
in a wide variety of programs
that might be missed by other testing strategies.

In order to illustrate PORT's usefulness,
we built a prototype compiler
for the language
and carried out a three part evaluation.
The first part consisted of using our prototype
to
re-implement the ``anomalies''
described in the earlier work on the SEA technique~\cite{DBLP:conf/issre/MooreCFW19}.
Side-by-side comparison shows that our new
descriptions are more concise,
readable,
and maintainable
than their original counterparts.
As an added benefit, by proving that
PORT can work on a wider variety of applications and activity formats, we can facilitate broader use of the SEA technique,
which has already proven effective as a bug detector.

Next,
we demonstrate the ease with which PORT can be extended to
other activity representations. 
PORT has features, discussed in
Section~\ref{SEC:architecture},
that allowed us to quickly
add support for recorded streams of USB
traffic. This new capability enabled us to test PORT's usefulness in detecting and simulating BADUSB-style attacks~\cite{badusb}.

Finally,
we conduct a performance evaluation
to demonstrate how quickly  PORT programs can
process recordings taken from real world network and compression applications.
Our results show that PORT programs are able to process large recordings in well under a second. We provide the scripts necessary to reproduce our performance figures along with our PORT implementation.


The main contributions in this work can be summarized as follows:

\begin{itemize}

\item We create a new domain specific language, {\em PORT},
  that allows for concise descriptions of patterns
  that may be
  found and transformed in an application's activity stream.

\item We show how PORT improves upon earlier work, such as that on the SEA technique, by offering a concise, but powerful way to write anomalies.

\item We demonstrate that PORT can be extended to other activity representations by using it to detect and simulate USB-based attacks and failures.

\item We provide an open source implementation of PORT available for immediate use
at: \textit{Link removed for blinding purposes.}

\end{itemize}


%%% Local Variables:
%%% mode: latex
%%% TeX-master: "paper"
%%% End:
