\section{Evaluation}
\label{sec:PORTevaluation}

Once we had an implementation of PORT, we designed a set of experiments to evaluate its  effectiveness in real world situations.
Specifically, we aim to answer the following questions:

\begin{itemize}

    % Re-implementing CS anomalies
  \item{Can PORT express the anomalies used by SEA to identify bugs?}

    % Extending to support RPC formats
  \item{How easy is it to extend PORT to support activity representations
    other than system calls?}

    % Dornhackl et al. defining maliciousness
  \item{What problems can be addressed by employing PORT on
  non-system-call activity representations?}

    % Performance information
  \item{Can PORT process input streams in a reasonable amount of time?}

\end{itemize}


\subsection{Expressing SEA Anomalies}
\label{sub:SEAAnomalies}
Given that this work is motivated
in large part
by a desire to expand the utility of the SEA technique,
our first experiment aims to reproduce the anomalies described
in~\cite{DBLP:conf/issre/MooreCFW19}.
Specifically,
we test PORT's ability to recreate
the study's unusual file type mutator,
and its cross-disk file move checkers, which were used to identify
the bulk of the bugs that were found.
% Our hope was that by using PORT we could eliminate boiler plate code,
% save effort by handling common tasks automatically, and improve reliability
% by providing a structured way to modify and output system call sequences.



\paragraph{Creating the Unusual File type Mutator.}
\label{subsub:UnusualFiletype}
For the first part of this experiment,
we used PORT to implement an ``unusual file type''
mutator.
As illustrated in Figure~\ref{lst:SEAListings},this mutator
takes an input trace
that contains a call to either {\tt stat()},
{\tt fstat()},
or {\tt lstat()}
and modifies its result data structure so
its {\tt ST\_MODE} member will indicate an unusual file type.
As can be seen in
Figure~\ref{lst:SEAListings}, this task can be expressed with only a few lines of PORT code.  In the figure,
lines 1 through 4 define what {\tt stat()}, {\tt fstat()}, and {\tt
lstat()} calls look like, and which parameter contains the result buffer.
Line 6 generates an accepting state that, when entered, produces an output
system call with a modified value in the return structure's {\tt st\_mode}
field.  The output can then be used to modify the results of a running
application's system calls in order to carry out the remaining steps of the
SEA technique.

\begin{figure}
\centering
\begin{lstlisting}[basicstyle=\ttfamily\scriptsize]
event fstat {filedesc: Numeric@0};
event stat {filename: String@0};
event open {filename: String@0, filedesc: Numeric@ret};

stat({filename: fn});
open({filename: ?fn, filedesc: fd});
fstat({fildesc: ?fd});
\end{lstlisting}
\begin{lstlisting}[basicstyle=\ttfamily\scriptsize]
event Statbuf {mode: String@2};
event anystat {stat sb: Statbuf@1}
        | {lstat sb: Statbuf@1}
        | {fstat sb: Statbuf@1};
anystat({sb: {mode: -> "st_mode=S_IFBLK"}});
\end{lstlisting}
\caption{The top listing contains a
PORT program that can detect whether a file was replaced as it was copied.
The bottom listing shows a program that
identifies a \texttt{stat}, \texttt{lstat}, or \texttt{fstat} call and modifies
  the \lstinline+ST\_MODE+ member of its \lstinline+statbuf+ output parameter to contain the value
  \lstinline+"S\_IFBLK"+. This indicates that the file being examined is a block device rather than a regular file.}
\label{lst:SEAListings}
\end{figure}

The original implementation of this mutator in~\cite{DBLP:conf/issre/MooreCFW19} consisted of 55 lines
of Python code, much of it error-prone state management code. While this code was needed, it harmed
readability and maintainability.
When compared to the original mutator, it became apparent that there were
several major advantages
to using
PORT:

\textit{Minimal boilerplate code:} Because PORT's capabilities are narrowly defined, it lacks the boilerplate
code associated with general purpose languages. No code is needed for
reading an input trace, managing mutator state, and producing output.
As a result,
functions can be generically implemented within PORT's core, eliminating
the need for users to do so manually.

\textit{No code required to filter out uninteresting calls:}
In PORT, there is no
need to explicitly exclude system
calls outside of the desired set.  Each statement defines a new state with
incoming and outgoing transitions configured to ignore any system calls not
dealt with in the PORT program.

\textit{Easy to modify call contents:}  PORT's operators make it
easy to change only those parameters  of a system call
 needed to produce output.
This is a far cry
from the Python program, which relies on manual and fragile string manipulation.

\paragraph{Supporting Cross-Disk Move Checkers.}

In the second part of this experiment we tested whether PORT can
implement the ``checkers'' used in SEA to determine if an
application can correctly move a file from one disk to another.
This task is a common source of bugs in Linux applications. As the
Linux
{\tt rename()} system call does not support moving files from one disk to
another,
applications must perform this complex
operation themselves.
Moore et al. identified the steps required to
correctly perform such a move by examining the source code of the ``mv''
command. The team then implemented a set of checkers to identify situations where an application does not carry out one
of these steps correctly.
In real world applications,
these checkers were able to identify bugs
in many popular applications and libraries that offer file movement
capabilities.

We evaluated each of the four checkers listed in Moore et al.'s work and
determined that PORT could implement three of them.
Figure~\ref{lst:SEAListings} shows our PORT implementation of
the ``File Replaced During Copy'' checker
that ensures {\tt fstat} is used after a file is opened but
before it is moved.  This pattern indicates that an application may be
storing the {\tt inode} number of the file -- a step that prevents a race
condition where the file is replaced during the move process.
Comparing it to the
original SEA checker,
which consisted of 45 lines of difficult to read and maintain Python code,
the PORT version is much more concise and its meaning is
less obscured by boilerplate and state management code.

This exercise did expose one of PORT's shortcomings.  Specifically,
we found that PORT cannot currently implement the ``Extended File
Attributes'' checker, which
 ensures that an application
preserves all of a file's extended attributes and re-applies them after the move.
PORT's lack of a list data structure made it difficult to create this checker as a
list is required to capture the values {\tt getxattr()}
and ensure they have all been applied with a
corresponding call to {\tt setxattr()}.
Though we are considering such a feature for future implementation,
we do not currently support it because such an extension could
hurt program clarity and make it harder to reason about
mutator behavior.


\subsection{Extending PORT to Other Activity Representations}
\begin{figure}
  \begin{lstlisting}[basicstyle=\ttfamily\small,xleftmargin=.8em]
event usbhid { src: String@0, dst: String@1,
               data: String@13, transfertype: String@10 };
num1 <- "00:00:1e:00:00:00:00:00";
num2 <- "00:00:1f:00:00:00:00:00";
src <- "2.1.1";
dst <- "host";
usbhid({src: ?src, dst: ?dst, data: ?num1}) -> usbhid({data: ->num2});
  \end{lstlisting}
  \caption{A demonstration PORT program that matches USB activity indicating the '1' key is being pressed and transforming it to a new frame where the '2' key is being pressed}
  \label{fig:USB}
\end{figure}


A key feature of PORT is the ease with which support for new activity representations
can be added. To demonstrate this
we implemented support for streams of USB activity. This format was chosen because of its ubiquity and its reliance on numerous parties correctly implementing a standard protocol.
Using PORT on streams of USB activity required implementing an appropriate transformer and developing some way to capture communications between USB devices.
For the latter, we used Wireshark because of its excellent traffic capture and dissection capabilities~\cite{wireshark}.
For the former, implementing such a transformer was a straightforward task taking only around three and a half.
Together, these two components allowed us to write PORT programs that could both identify patterns and transform streams of USB activity in minutes.

\textbf{BADUSB} As one test scenario, we settled upon the recent type of USB-based attack known as BADUSB~\cite{badusb}.
These attacks utilize small USB devices that resemble thumb drives. However, rather than storing files, when plugged into a targeted computer, these devices register themselves as human interface devices. They can then rapidly send keystrokes to execute malicious commands before a human is able to react.
Our goal was to construct PORT programs to recognize these attacks within a recording of a machine’s USB traffic, and simulate them by transforming an innocent USB recording into one containing such an attack.
The latter could be replayed using a device similar to those used in actual attacks in order to assess whether or not a computer’s defensive measures (e.g. antivirus software or specialized anti-BADUSB programs) are able detect the attacks.

The first PORT program we wrote detects a USB device attempting to execute
powershell in a mode where its security policy is bypassed.
This is a common starting point for BADUSB attacks that seek to execute complex payloads, such as powershell scripts.
The program we wrote detects USB frames that contain a sequence of ``scan codes'' in which a series of keystrokes spell out ``powershell -Exec bypass.''
Detecting this string is critical because it explicitly disables security controls, a step that should only be taken under special circumstances.
We were able to use this program to detect the target sequence in streams of USB traffic recorded from a real computer using a standard USB keyboard.  An abbreviated example program that performs this sort of operation is shown in Figure~\ref{fig:USB}.

A more advanced example of PORT's capabilities with USB streams involves simulating a BADUSB attack.
In a similar fashion to our detection program, this program identifies USB human interface device frames, and then transforms the scan codes they contain to yield key presses that spell out ``powershell -Exec bypass.''
An analogous operation is illustrated in the latter half of Figure~\ref{fig:USB}.
In this way, the program can use an innocent stream to create a malicious one capable of driving a BADUSB attack. In doing so, a user can determine how well a system's defenses can detect and prevent such a scenario.

\textbf{Device ID Conflicts} Our second test involved simulating a USB device receiving an inappropriate ``vendor ID'' or ``product ID'' from its manufacturer.
Because these identifiers help determine which driver to load for a device,
incorrect settings can, at best, cause the device to not work correctly~\cite{wrongid}.
In other cases,
the malfunction is problematic enough that kernel developers block these devices from operation to prevent further problems~\cite{barscannerbug}.
We used PORT to simulate a manufacturer that has reused device identifiers across multiple devices.
To do so, we wrote a PORT program that monitors a stream of USB traffic for USB device registrations.
The first time it encounters one, the program stores the \textit{vendorID} and \textit{productID} field into registers.
When subsequent registrations are encountered,
their identifiers are rewritten using these stored values.
As a consequence, the action produces a new stream of USB activity in which many devices share incorrect device identifiers. The new stream can then be used to test a system's response to such a mis-configuration.

The above successes show that PORT is easily capable of working with activity representations beyond system calls.
We did, however, identify one point of friction -- keyboard scan codes are cumbersome to work with.
Future work could employ meta-programming to smooth the rough edges of lower level activity representations.
This minor complication notwithstanding,
our ability to recognize patterns and transform USB streams demonstrates
that PORT is able to take
the SEA technique from one domain and use it successfully in another one.


\subsection{PORT's Performance}

%It doesn't matter how useful a tool may be
%if it takes too long to complete its work.
%Though our implementation is
%only a prototype, we wanted to make sure that its performance was not
%overly slow.
Our final experiment evaluates the
time required for our current implementation
to identify specific
patterns within real-world system call traces.
To get a realistic set of test traces
we chose
eight widely used network applications
and four popular compression utilities that
offered a sufficient level of complexity.

We recorded test traces using the following
experimental setup. Five of the applications are clients that operate by connecting to an appropriate service. They were recorded as they made this connection and completed a small request (e.g. transmitting or receiving a file).
Three of the applications were servers, and were recorded as they accepted a connection from an appropriate client and serviced a small request.
The compression utilities were recorded as they compressed a file or decompressed an archive.
These recordings were made with {\tt strace}
and then processed using a PORT program.
For the network applications the program
identifies the sequence of system calls that implement
a client or server's request handling loop.
The compression utility recordings were processed using a separate
program that finds the read/write loop responsible for
carrying out a compression or decompression operation\footnote{Recordings are pre-processed to remove system calls
related to executable loading and process creation.}.  Table~\ref{tbl:RealWorldPerformance}
shows the average time required to complete the specified operation
based on one hundred executions, as well as the number of system calls being processed in the recording.  This performance evaluation was run on a laptop using a four core processor running at 3.4 ghz with 16 gigabytes of memory.
Our PORT compiler comes with a script to reproduce these results with one command.

\begin{figure}[t]
\centering
\small
\resizebox{\linewidth}{!}{
  \begin{tabular}{|c|l|l}
Utility and Operation & Exec. Time & No. Syscalls \\
              \hline
gzip compress file   & 0.110           & 17  \\
gzip decompress file & 0.107           & 35  \\
rar compress file    & 0.112           & 109 \\
rar decompress file  & 0.109           & 87  \\
bzip decompress file & 0.102           & 25  \\
ncat server          & 0.103           & 43  \\
socat server         & 0.108           & 71  \\
http.server server   & 0.114           & 21  \\
rsync client         & 0.132           & 274 \\
ssh client           & 0.159           & 850 \\
ftp client           & 0.160           & 891 \\
scp client           & 0.135           & 490 \\
telnet client        & 0.106           & 23  \\
\hline
BADUSB               & 0.111           & 1116 lines \\
ID Conflict          & 0.117           & 18992 lines \\
\end{tabular}
  }
\caption{Average time required to process the specified recording based on 100 executions.}
\label{tbl:RealWorldPerformance}
\end{figure}


The results in Table~\ref{tbl:RealWorldPerformance} show that
PORT's
processing time increases in line with the total number of system calls
in the recording.  We anticipate that much of this processing cost is
associated with setting up the Python execution environment and that a more
optimized implementation could improve performance gains in this area.
Further,
it is likely that PORT's performance is closely tied to
disk throughput,
and that advancing the transducer
as each system call is evaluated
adds little additional overhead.


\subsection{Threats to Validity}

While we conducted this evaluation as rigorously as possible,
there are a few areas where some ambiguity may exist.
First, in our work with USB activity, we limited ourselves to US English
keyboards.  Other keyboard languages and designs may require enhancements to our transformer or programs.  Additionally, our performance evaluation samples from only a handful
of programs that were selected by popularity rather than at random.
% This was done due to the selected program's popularity and the lack of time to complete an exhaustive exploration of all candidates.
There may be programs that would diverge from the performance trend we report above. Future work can determine how widely this phenomena occurs and if any subsequent modifications are required.

%%% Local Variables:
%%% mode: latex
%%% TeX-master: "paper"
%%% End:
