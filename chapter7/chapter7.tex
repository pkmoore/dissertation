\chapter{Conclusion}
\label{chap:conclusion}

Environmental bugs can cause applications to fail after they have deployed, which necessitates costly efforts to diagnose,
fix,
and deploy them.
In this work we present a novel approach for identifying such bugs \textit{before} deployment.
These efforts center around the SEA technique and its implementation,
CrashSimulator.
This tool uses SEA to identify bugs by simulating anomalies using recordings of the system calls an application makes.
Using CrashSimulator,
we were able to find many high-impact bugs in popular, battle-tested applications.

We also discussed PORT,
our domain specific language for describing anomalies and opportunities to simulate them.
Our goal with PORT was to develop a way to ``capture'' anomalies that was easy and efficient
This was possible because PORT is closely tailored toward its task of
finding and manipulating patterns in application activity recordings.
This narrow purpose allowed us do build a language that
helps users write clear,
maintainable programs
by automatically handling
aspects of the task that are not directly related to their goal.
We were able to use PORT to replace existing CrashSimulator checkers and mutators.
Further, PORT allowed us to expand SEA to other domains, such as anomalies that appear in USB traffic.
This capability allowed us to write programs that can simulate malicious traffic, such as BADUSB attacks
and failure-causing device configurations like reused device identifiers.

Finally,
we conducted a user study where participants used CrashSimulator to find and fix environmental bugs in popular applications.
Our participants had success finding bugs and writing patches to correct them but met resistance getting these patches accepted by project maintainers.
This resistance came down to improper patch submission on the part of our participants and a reluctance on the part of maintainers to agree that the identified mis-behaviors were bugs.
Moving forward, the former could be corrected by better educating participants on patch submission etiquette. The latter will require an educational effort around environmental bugs and their impact on software. If the consequences of leaving these bugs at large was more clearly understood, we feel this resistance will be reduced.

We hope that having tools like CrashSimulator and PORT will make such an effort possible.
Additionally, 
we hope to see wide-spread adoption of these
tools in continuous
integration and deployment pipelines. This could enable current and future
applications to be tested against an ever-increasing repository of
environmental bugs.
These efforts would all benefit from the development of a user community in which developers can exchange PORT programs and their experiences. The body of knowledge that would result from such a communal effort will help to increase understanding of the situations that caused failures.  Collectively, 
this could enable the introduction of new applications without the catastrophic consequences of a deployment failure. 