\chapter{Conclusion}
\label{chap:conclusion}

Environmental bugs can cause applications to fail after they have deployed necessitating costly efforts to diagnose,
fix,
and deploy an application.
In this work we have covered our novel approach for identifying such bugs \textit{before} deployment.
These efforts center around the SEA technique and its implementation,
CrashSimulator.
This tool uses SEA to identify bugs by simulating anomalies using recordings of the system calls an application makes.
Using CrashSimulator,
we were able to find many high-impact bugs in popular, battle-tested applications.

We also discussed PORT,
our domain specific language for describing anomalies and opportunities to simulate them.
PORT was inspired by a CrashSimulator user study which found that ``capturing'' anomalies using existing
general purpose programming languages was awkward and error prone.
PORT is closely tailored toward its task of encoding anomalies which helps its users write clear,
maintainable programs that encode complex anomalies.
We were able to use PORT to write programs that could replace existing CrashSimulator checkers and mutators.
Further, PORT allowed us to expand SEA to other domains such anomalies that appear in USB traffic.
This capability allowed us to write programs that can simulate malicious traffic such as BADUSB attacks
and failure-causing device configurations like reused device identifiers.