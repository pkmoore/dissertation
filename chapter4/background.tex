\section{Background and Motivation}
\label{SEC:background}


\subsection{Our Motivating Example}
\label{sec:MotivatingExample}
% How SEA showed success using system calls

The initial impetus for this work was sparked by the efforts of Moore et al.~\cite{DBLP:conf/issre/MooreCFW19}
that lead to the creation of
the Simulating Environmental Anomalies (SEA) technique.
This effort centered on the key insight
that problematic
environmental properties,
known as anomalies, are visible in the
communications between the components that make up an application.
The researchers found that,
once captured,
these anomalies
could be
used to simulate and 
test
an application as if
it had been encountered
in the real world.
In that work, results of system calls made
during execution were recorded, modified, and replayed to simulate whether or not the application
responded correctly to the anomaly.
Using this strategy, the authors were able to identify a number of bugs
in major applications.

As a concrete example of the above
consider the ``Unusual File types'' anomaly
discussed in the SEA paper's evaluation.
This anomaly may be problematic
when an application running under Linux
attempts to open and read data from a file on disk.
In addition to ``regular'' files,
which are used to store text and
binary data,
Linux supports several special file types.
Writing to and
reading from each of these file types requires special procedures.
As a result, an application should check the type of files it intends to
process so that it can handle it correctly.
The SEA technique
may simulate the presence of such a file by
modifying
the return value
of a {\tt stat} call.
Moore et al. accomplish this manually
by writing a script
for each anomaly
to process system call traces.
PORT improves on this process by
making it easier to describe
where and how to make required modifications.

Our takeaway
from reading this study
was that an application's activity
can be systematically mined to find bugs.
The work described in this paper
shows that the best way to extract this data
is to treat application activity
as a sequence of events,
thus allowing us to analyze it using
proven event processing techniques. 
Yet, it became apparent that existing
stream processing tools
are not well suited to this task. We needed a 
novel tool to leverage
SEA's success beyond {\tt strace} and system calls to encompass other activity types,
such as calls to library functions
and remote procedure calls.
This could be achieved by operating in a fashion agnostic to the way an application's activity has been recorded. Such a tool would require a distinct language. 


%%% What are we really doing here.  WHY did we decided to make this language?
%%% That's really the question that needs to be answered.  I think that instead
%%% of making things up that we had the language and decided to apply it here
%%% we need to talk about how seeing the success of this work motivated us to
%%% try it at a larger scale.  We can avoid talk of "Augmenting" SEA or
%%% whatever though

%\subsection{Explaining Environmental Bugs}
%
%Before we discuss the details of our bug-finding efforts it is necessary to
%draw a boundary around the types of bugs we are targeting.
%This work sets its sights on the bugs that occur when some external
%entity supplies unexpected or incorrect data to an
%application resulting in its failure.
%Such bugs may appear in simple situations like a library function that
%returns data in an unexpected format or in scenarios as complex
%as a request to a remote
%system returning bad data because of corruption on an intermediate network
%node.
%Both cases share the commonality that the symptoms of misbehavior
%can be found in a recording of the application's activity.
%These bugs, defined by Moore et al. as ``environmental bugs,''
%often cause applications to fail after deployment
%and occur with such frequency
%that the ``works on my machine'' phenomenon is a well known
%source of pain
%and frequent topic of discussion
%in software and project management
%literature~\cite{worksonmymachine}.
%Post-deployment bugs are a widespread problem
%and they come with significant costs
%both in financial terms and,
%in the worst cases,
%loss of life~\cite{WONG201768}.
%
%The impact of these bugs continues to be reinforced by the regular
%appearance of dangerous environmental bugs in major pieces of
%software~\cite{devzeroroot}.  And it appears that no class of application
%is safe with environmental bugs affecting operating
%systems~\cite{ubuntuappaport},
%user applications~\cite{westerndigitalsymlink} and crucial security
%programs~\cite{sudocopy} in the
%last year alone!
%
\begin{figure}
\begin{minipage}{.5\textwidth}
  \begin{lstlisting}[gobble=4]
    # Before Modification
    open("example.txt", O_RDWR, 0) = 3
    fstat(3, {...}) = 0
    getpid() = 34355
    write(3, "Hello World!\n", 13) = 13
    close(3) = 0
  \end{lstlisting}
 \end{minipage}%
 \begin{minipage}{.5\textwidth}
  \begin{lstlisting}
    # After Modification
    open("example.txt", O_RDWR, 0) = 3
    fstat(3, {...}) = 0
    getpid() = 34355
    write(3, "Hello World!\n", 13) = 13
    close(3) = -1
  \end{lstlisting}
  \end{minipage}
  \caption{A system call trace of a program that opens, writes to, and closes
  a file before and after modifying the return value of the \texttt{close()} call.}
  \label{fig:StraceListing}
\end{figure}

\subsection{Why a New Domain Specific Language?}
\begin{figure}
\centering
\begin{minipage}{.5\textwidth}
  \begin{lstlisting}[basicstyle=\small,gobble=4,xleftmargin=.8em]
    event open {filedesc: Number@ret};
    event read {filedesc: Number@0};
    event close {filedesc: Number@0,
                 retval: Number@ret};
    open({filedesc: fd});
    not read({filedesc: ?fd});
    close({filedesc: ?fd});
  \end{lstlisting}
  \end{minipage}%
  \begin{minipage}{.5\textwidth}
  \begin{lstlisting}[basicstyle=\tiny]
pattern = Pattern(
  SeqOperator(
     PrimitiveEventStructure("OPEN", "a"),
     NegationOperator(
        PrimitiveEventStructure("READ","b")
     ),
     PrimitiveEventStructure("CLOSE","c")),
AndCondition(EqCondition(
     Variable("a", lambda x: x["File Handle"]),
     Variable("c", lambda x: x["File Handle"])),
     EqCondition(
     Variable("c", lambda x: x["File Handle"]),
     Variable("b", lambda x: x["File Handle"]))),
timedelta(minutes=10))
  \end{lstlisting}
  \end{minipage}
  \caption{
  The left listing is a PORT program that finds situations
  where a program opens a file and closes it without reading from it.
  The right listing is an OpenCEP listing that identifies the same pattern.
}
  \label{fig:CompareListing}
\end{figure}


The decision to create a new domain specific language was not one we
undertook lightly. 
as such an effort takes
a significant amount of work
to define,
implement, document, and support. Yet, as we detail in this section, existing systems could not deliver the features we needed. For starters, the language  had to be able to identify specific
patterns as they appear in a recording of application activity.
At first glance, it seems a simple model,
such as a deterministic finite automaton (i.e. a DFA or a FSA), would suffice.
The description could be written
in a language similar to
regular expressions, but enhanced to operate on complex
structures rather than symbols in a string.
Unfortunately,the solution is no so simple due to  dependencies
between activities in these patterns
across time,
as shown in Figure~\ref{fig:StraceListing},
The {\tt open()} call produces a file descriptor that a subsequent {\tt
read()} call may match.
Yet, a standard DFA or FSA cannot match patterns with this sort of dependency. 
Instead, we need a language
that can easily capture
the internal contents of events,
like argument data,
pointer addresses,
and return values that can be
manipulated for
reuse in subsequent operations.

One possibility is to deploy more expressive automata models, such as register automata~\cite{DBLP:journals/tcs/KaminskiF94} or session automata~\cite{DBLP:journals/corr/BolligHLM14}. Yet, these models cannot produce a modified output and, as the 
SEA researchers found,  the ability to
{\textit modify} activity allows researchers to create scenarios that ensure
application failures rather than  waiting for them to possibly occur.
Several feature-rich event processing
languages and libraries do have these capabilities, but
modifying and outputting
incoming events
is by no means a straightforward experience.
In many cases, producing such an output stream  would require
falling back on the fully-featured nature of a host language (e.g. Java) -- a situation
we hoped to avoid.

In the initial stages of PORT's development, we also
evaluated several
complex event processing (CEP) languages that
could
provide some of the pattern and predicate matching primitives
that we wished to
incorporate.
Sadly,
these languages did not
support  the features
we require,
or were too complex
for the easy to use  system we wanted to offer.
Typically,
programs for these complex event processing engines are
written in the engine's build or host language,
such as Java,
Scala or
Python.
Such languages generally bring with them a great deal of boilerplate code,
that can obscure or confuse
the program's meaning.
Recent studies
have affirmed that excessive and complicated code
patterns can harm
understanding and
maintainability~\cite{misunderstandings}.
Further,
it means that the author,
and future maintainers,
of a
program must be fluent in this host language.
Finally,
these languages include features and optimizations
that would likely not be useful for our application domain, such as time-based event windows,
merging multiple event streams,
and the calculation of
summary statistics over specified fields. 


To demonstrate these points concretely,
we consider two programs that recognize a system call sequence wherein an
application opens and closes a file
without reading from it.
Figure~\ref{fig:CompareListing} shows a PORT program that implements this
task.  The first three statements in this program define the events (i.e.
system calls) and parameters relevant to the task it performs.  The final three statements express that the event stream
should contain an \texttt{open()} call followed by a \texttt{close()} call on the same file descriptor, with no intermittent \texttt{read()} call.

We compare this program 
\footnote{which can be executed using the code and tools available at: \textit{Link Removed for Blinding Purposes}}
with a corresponding program written in OpenCEP~\cite{open_cep_website},
a CEP module for the Python programming language.
As shown in Figure~\ref{fig:CompareListing},
the PORT program is both more concise with its main work being done in just
three lines of code, and more readable because it is not constrained by the
requirements of a host programming language.
Unlike  OpenCEP and similar languages, PORT can modify and output an event.
That is, the program in Figure~\ref{fig:CompareListing} could be
re-purposed to not only find the appropriate pattern but also modify the
results of the {\tt close} call to simulate a failure.
This would be achieved
by editing line 8 of the program to contain an output
clause as follows:
\begin{lstlisting}[numbers=none,xleftmargin=0em,gobble=2,columns=strict]
  close({filedesc: ?fd}) -> close({retval: -1});
\end{lstlisting}
If the modified PORT program is executed on the system call trace shown in the upper half of Fig.~\ref{fig:StraceListing}, it will produce the modified system call trace shown in the bottom half.

% A domain specific language like PORT provides a more
% concise syntax that makes it easier for all individuals to read and
% write relevant programs in.
% In light of this, we believe the benefits of a
% new programming language
% focused on letting its users get a lot of work done
% with a small amount of easily-readable code should be self-evident.

% The second front involves PORT's programming paradigm.
% While other event processing languages tend toward functional or
% declarative programming,
% PORT programs more closely follow an imperative programming style.
% We came to this decision because studies
% have shown that developers are more likely to be familiar and comfortable
% with such a paradigm~\cite{XXXX}.  We believe this will make it easier for
% developers to learn the language, foster greater popularity, and it aligns
% with the goals presented in our motivating example.




%%% Local Variables:
%%% mode: latex
%%% TeX-master: "paper"
%%% End:
