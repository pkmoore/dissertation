%\vspace*{3ex plus 0.6fil}
\begin{center}
\addcontentsline{toc}{chapter}{Abstract}

{\large\bf
   AN ABSTRACT\\[3ex]
   A PORT in Stormy SEAs: Leveraging Past Problems to Prevent Future Failures\\[2ex]
   by\\[3ex]
   Preston K.\ Moore\\[3ex]
   %% Advisor name
   Advisor: Justin\ Cappos\\[2ex]
   %% Co-advisor, or comment out
   Co-Advisor: Phyllis Frankl
}\\[3ex]
Submitted in Partial Fulfillment of the Requirements\\[2ex]
for the Degree of Doctor of Philosophy (Computer Science)\\[3ex]
% The date appearing on the title page should be the month and year of
% the expected degree award (e.g., January 20XX or May/June 20XX)
% and not the completion date of the work.
May 2022
\end{center}

\vspace*{2.5ex}

%Paragraph 1
%%Software repositories, or servers that host and distribute
%%software updates, are becoming increasingly important in a wide variety of
%%settings,

Software is increasingly expected to operate across widely varying environments.
If not handled appropriately, any
differences between these environments
could cause even well-tested applications to fail upon deployment.
Anticipating and being prepared to handle all the possible combinations of
software and hardware that can affect an application’s ability to operate
is a major challenge.
To make matters worse,
recovering from post-deployment failures is expensive,
harmful to user experience,
and damaging to developer reputations.

What is needed is a way to detect situations where an application may fail
\textit{before} it is deployed so that its deficiencies can be corrected without the negative consequences of a crash.
We observed that
the causes of these environment-related failures
often can be seen in the
communications between an application and its environment.
These communications can include the
system calls the application makes or the messages it sends across a network.
By taking advantage of this concept,
we were able to develop two techniques that allow developers to catch environmental bugs
before an application is deployed. 

The first technique,
known as Simulating Environmental Anomalies (SEA),
allows developers to record features from one environment in which
an application has failed and use them to determine if other
applications are likely to fail in a similar manner.
Using this technique enabled us to find many high impact bugs involving mishandling of unusual files,
moving files across disks,
and inappropriate network timeouts
in popular,
battle tested applications.

The second technique is PORT, 
a new  domain specific programming language
that uses event processing techniques
to simplify procedures for recording and simulating an anomaly.
PORT
allows developers
to quickly write expressive programs that can analyze a stream of communications between an application and  its environment.
PORT programs can both detect opportunities to
simulate an anomaly, \textit{and} modify the stream so that anomaly is present.

We tested how well PORT achieved its intended purpose by using it to recreate and
replace the anomalies used in our original SEA work.
These new PORT programs were shorter, easier to read, and simpler to
maintain than the lengthy and complex Python scripts they replaced.
We also took advantage of PORT's extensibility to apply SEA to USB
traffic.
As SEA enabled us to write programs that could identify patterns in, and
modify recordings of, USB communications,
we were able to simulate
BADUSB style attacks by modifying innocent streams
so that they contained the harmful features of such an attack.
We were also able to simulate device identifier conflicts -- a situation that has
long plagued operating system developers -- by using PORT to modify a recording so  these conflicts
would be present in device registration frames.
In both cases, these recordings could then be used
to test how a system might respond should they occur in the real world.

%Paragraph 4
Our hope in developing these tools and techniques was to see them widely
used in improving application reliability and security.
But, to do so, these tools needed to be accurate, efficient, and easy to use. To test the latter quality, we 
conducted a user study to see how well developers could integrate SEA into their workflows.
At the conclusion of the study, we found that participants were able to find new bugs in widely deployed software,
build corrective patches,
and submit them to the appropriate upstream maintainers.

Unfortunately,
in spite of some success finding and fixing bugs,
our developers met resistance in getting their fixes incorporated.
This resistance sprang from two sources: improper patch submission on the part of our participants, and a reluctance on the part of maintainers to view certain misbehaviors as bugs.
The former may be dealt with by improving how we teach novices to interact with maintainers. The latter
will require an effort to educate project maintainers on the value of identifying and fixing environmental bugs. Having reliable tools and techniques that can  simplify the implementation of these processes should make these educational efforts easier to promote. 


\vspace*{3ex plus 1fil}
