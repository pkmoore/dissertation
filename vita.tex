\prefacesection{Vita}


\textbf{Preston Kent Moore} was born in Bristol, Tennessee on November 18, 1988.  He received a Bachelor of Science degree in Computer Science from East Tennessee State University (ETSU) in 2011.
Following this, he held a System Analyst position at Eastman Chemical Company for one year
before returning to ETSU to complete a Master of Science degree in Computer Science.
Preston stayed on at the school as a Lecturer in the Computer Science Department, and a Software Project Manager position with the school's Emerging Technology Center before moving to New York City to pursue a doctoral degree.
In the fall of 2015 he began work on his Ph.D. at NYU Tandon School of Engineering's Computer Science and Engineering Department. During his tenure there, he worked as a research assistant in the Secure Systems Lab, and also held two adjunct instructor positions in the CSE department, teaching Introduction to Problem Solving and Computer Security.

Preston's research focused on application security and reliability, specifically
exploring a novel testing technique that exposed applications to simulations of scenarios that caused other applications to fail.
To investigate this topic, he developed two new techniques: SEA and PORT.
SEA finds software bugs by capturing and simulating the differences between the environments in which they will run.
PORT is both a technique and its associated domain specific language that applies stream and event processing strategies to identifying and transforming  sequences of application activity within recordings.

The output of  this work is captured in two papers. The first, covering SEA, was published as a conference paper at the International Symposium on Software Reliability and Engineering (ISSRE).
It was presented in Berlin, Germany at ISSRE 2019 and received the Best Paper and Presentation award.
The second was presented as a poster at RSA 2019 as part of the conference's security scholar program.
A full-length paper covering PORT is currently under consideration by the International Conference on Software Technologies.

Outside of the confines of the lab, Preston has also conducted research on improving the way information security concepts are taught to novices.
As part of this work he developed and presented a session for high school students where card magic was used to scaffold unfamiliar concepts.
Observations and insights from this work were published by the Consortium for Computer Science in Colleges (CCSC), and also presented in a poster format at the 2022 Association for Computing Machinery's Special Interest Group for Computer Science Education Conference.


%% From the rules:
% Give date and place of birth and a brief educational and
% professional history.  Clearly state period of time devoted to the
% research or project, the laboratories in which it was performed, and
% the source of any special support (research contract, research
% grant, fellowship, assistantship, traineeship, etc.). A vita page is
% NOT the same thing as a resume.